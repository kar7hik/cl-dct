% Created 2020-05-06 Wed 19:34
% Intended LaTeX compiler: pdflatex
\documentclass[11pt]{article}
\usepackage[utf8]{inputenc}
\usepackage[T1]{fontenc}
\usepackage{graphicx}
\usepackage{grffile}
\usepackage{longtable}
\usepackage{wrapfig}
\usepackage{rotating}
\usepackage[normalem]{ulem}
\usepackage{amsmath}
\usepackage{textcomp}
\usepackage{amssymb}
\usepackage{capt-of}
\usepackage{hyperref}
\usepackage{minted}
\author{Karthik Kumar}
\date{}
\title{Discrete Cosine Transformation}
\hypersetup{
 pdfauthor={Karthik Kumar},
 pdftitle={Discrete Cosine Transformation},
 pdfkeywords={},
 pdfsubject={},
 pdfcreator={Emacs 26.3 (Org mode 9.1.4)}, 
 pdflang={English}}
\begin{document}

\maketitle

\section{Discrete Cosine Transformation:}
\label{sec:org906b88f}

\begin{equation}
\tilde{X}^{c2}[k] = \sqrt{\frac{2}{N}} \tilde{\beta}[k]\sum_{n = 0}^{N - 1} x[n] \cos \left(\frac{\pi k \left(2n + 1\right)}{2N}\right), \qquad k = 0, 1,..., N - 1.

where, 
\tilde{\beta}[k] = 
        \begin{cases}
                \frac{1}{\sqrt{2}}, & \quad k = 0, \\
                1, & \quad k = 1, 2,..., N - 1. 
        \end{cases} \\

With scaling factor, \\

\tilde{X}^{c2}[k] = {\omega}[k]\sum_{n = 0}^{N - 1} x[n] \cos \left(\frac{\pi k \left(2n + 1\right)}{2N}\right), \qquad k = 0, 1,..., N - 1. \\

where,

{\omega}[k] = 
	\begin{cases}
		\frac{1}{\sqrt{N}}, & \quad k = 0, \\
		\\
		\sqrt\frac{2}{N}, & \quad k = 1, 2,..., N - 1. 
	\end{cases}
\end{equation}


\section{Inverse Discrete Cosine Transform:}
\label{sec:orgd608c84}

\begin{equation} 

x[n] = \sqrt\frac{2}{N}\:\sum_{n = 0}^{N-1}\:\tilde\beta[k]\:\tilde{X^{c2}}\:\cos\left(\frac{\pi k\left(2n+1\right)}{2N}\right), \qquad 0 \leq n \leq N-1, \\

where,

\tilde{\beta}[k] = 
	\begin{cases}
		\frac{1}{\sqrt{2}}, & \quad k = 0, \\
		1, & \quad k = 1, 2,..., N - 1. 
	\end{cases} \\

With scaling factor, \\

x[n] = {\omega}[k]\:\sum_{n = 0}^{N - 1}\: \tilde{X}^{c2}[k]\:\cos \left(\frac{\pi k \left(2n + 1\right)}{2N}\right), \qquad 0 \leq n \leq N-1, \\

where, 

{\omega}[k] = 
	\begin{cases}
		\frac{1}{\sqrt{N}}, & \quad k = 0, \\
		\\
		\sqrt\frac{2}{N}, & \quad k = 1, 2,..., N - 1\ 
	\end{cases}
\end{equation}
\end{document}
